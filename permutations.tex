\documentclass[11pt, oneside]{article}   	% use "amsart" instead of "article" for AMSLaTeX format
\usepackage{geometry}
\geometry{
    a4paper,
    total={160mm,250mm},
    left=20mm,
    top=20mm,
}
%\usepackage[parfill]{parskip}    		% Activate to begin paragraphs with an empty line rather than an indent
\usepackage{graphicx}   % Use pdf, png, jpg, or eps§ with pdflatex; use eps in DVI mode
						% TeX will automatically convert eps --> pdf in pdflatex		

\usepackage{amsmath,amsfonts,amssymb}
\usepackage{tensor}
\usepackage{bm}

%SetFonts


\title{Levi-Civita-Symbol a.k.a. epsilon symbol}
\author{Michael Gehricke}
%\date{}							% Activate to display a given date or no date

\begin{document}
\maketitle
%\section{}
%\subsection{}
\newcommand{\vect}[1]   {\mathbf{#1}}
\newcommand{\tenmix}[3] {{#1}^{#2}_{#3}}
\newcommand{\omm}[2]    {\bm{\sigma}_{#1}\;\bm{\sigma}_{#2}}
%\equiv
Flipping the order of two indices in a permutation flips its sign. Doing
it twice is equal to no flip. In a 3-number sequence a double flip is equal
to a rotatation of the sequence, no matter wether rotated left or right.
So if we have a sequence 1234, rotating the last three digits keeps the sign.
We can do that two times, the third rotation gives us the original sequence.
In order to get all odd permutations we flip the second and the third digit to
get 1324 and rotate the last three digits to get the two remaining sequences.
Then all rotations of 4 digits form the complete set of sequences.
\\ \\
Permutations of $\epsilon_{ijkl}$ that are not zero.
\begin{equation}
    \begin{matrix}
          1  &   1  &   1  &&  -1  &  -1  &  -1  \\
        1234 & 1342 & 1423 && 1324 & 1243 & 1432 \\
        2341 & 2134 & 2314 && 2413 & 2431 & 2143 \\
        3412 & 3421 & 3142 && 3241 & 3124 & 3214 \\
        4123 & 4213 & 4231 && 4132 & 4312 & 4321
    \end{matrix}
\end{equation}
$\bm{\sigma}$-matrices:
\begin{equation}
\bm{\sigma}_1 \equiv \begin{bmatrix} 1 & 0 \\ 0 & 1 \end{bmatrix} \quad
\bm{\sigma}_2 \equiv \begin{bmatrix} 0 & -1 \\ 1 & 0 \end{bmatrix} \quad 
\bm{\sigma}_3 \equiv \begin{bmatrix} -i & 0 \\ 0 & i \end{bmatrix} \quad
\bm{\sigma}_4 \equiv \begin{bmatrix} 0 & -i \\ -i & 0 \end{bmatrix}
\end{equation} \\
Here is the multiplication table for the $\bm{\sigma}$-matrices:
\begin{equation} \bm{\sigma}_i \bm{\sigma}_j =
    \begin{pmatrix}
        \bm{\sigma}_1 &\bm{\sigma}_2 &  \bm{\sigma}_3 &  \bm{\sigma}_4 \\
        \bm{\sigma}_2 &-\bm{\sigma}_1 &  \bm{\sigma}_4 & -\bm{\sigma}_3 \\
        \bm{\sigma}_3 &-\bm{\sigma}_4 & -\bm{\sigma}_1 &  \bm{\sigma}_2 \\
        \bm{\sigma}_4 &\bm{\sigma}_3 & -\bm{\sigma}_2 & -\bm{\sigma}_1 \\
    \end{pmatrix}
\end{equation}
\begin{equation}
    \begin{aligned}
        \Sigma_{ij}
        &= \epsilon_{ijkl} \; \bm{\sigma}_k \bm{\sigma}_l \\
        &=
        \begin{pmatrix}
            0                     &  \omm{3}{4}-\omm{4}{3}  &  \omm{4}{2}-\omm{2}{4}  &  \omm{2}{3}-\omm{3}{2} \\
            \omm{3}{4}-\omm{4}{3} &  0                      &  \omm{4}{1}+\omm{1}{4}  &-(\omm{1}{3}+\omm{3}{1}) \\
            \omm{4}{2}-\omm{2}{4} &-(\omm{4}{1}+\omm{1}{4}) &  0                      &  \omm{1}{2}+\omm{2}{1} \\
            \omm{2}{3}-\omm{3}{2} &  \omm{1}{3}+\omm{3}{1}  &-(\omm{1}{2}+\omm{2}{1}) &  0
        \end{pmatrix} \\
        &=
        \begin{pmatrix}
            0              &  2\bm{\sigma}_2  &  2\bm{\sigma}_3  &  2\bm{\sigma}_4 \\
            2\bm{\sigma}_2 &  0               &  2\bm{\sigma}_4  & -2\bm{\sigma}_3 \\
            2\bm{\sigma}_3 & -2\bm{\sigma}_4  &  0               &  2\bm{\sigma}_2 \\
            2\bm{\sigma}_4 &  2\bm{\sigma}_3  & -2\bm{\sigma}_2  &  0
        \end{pmatrix} \\
        &= \frac{1}{2} \bm{\sigma}_k \bm{\sigma}_l - 
        \begin{pmatrix}
            -1 & 0  & 0  & 0 \\
             0 & 1  & 0  & 0 \\
             0 & 0  & 1  & 0 \\
             0 & 0  & 0  & 1
        \end{pmatrix} \bm{\sigma}_1 \\
    \end{aligned}
\end{equation}
%
We obtain an interesting result:
\begin{equation} \bm{\sigma}_i \bm{\sigma}_j = 
    2 \; \epsilon_{ijkl} \; \bm{\sigma}_k \bm{\sigma}_l 
    +
    \begin{pmatrix}
        -1 & 0  & 0  & 0 \\
         0 & 1  & 0  & 0 \\
         0 & 0  & 1  & 0 \\
         0 & 0  & 0  & 1
    \end{pmatrix} \bm{\sigma}_1 \\
\end{equation}
%
For the actual product of two arbitrary elements of $U(2)$ we get
\begin{equation}
    a^i\bm{\sigma}_i b^j\bm{\sigma}_j = 
    a^i a^j
    \begin{bmatrix}
        2 \; \epsilon_{ijkl} \; \bm{\sigma}_k \bm{\sigma}_l +
        \begin{pmatrix}
            -1 & 0  & 0  & 0 \\
             0 & 1  & 0  & 0 \\
             0 & 0  & 1  & 0 \\
             0 & 0  & 0  & 1
        \end{pmatrix} \bm{\sigma}_1 \\
    \end{bmatrix}
\end{equation}

\section{Delta Systems}




\end{document}