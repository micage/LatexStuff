\documentclass[11pt, oneside]{article}   	% use "amsart" instead of "article" for AMSLaTeX format
\usepackage{geometry}                		% See geometry.pdf to learn the layout options. There are lots.
\geometry{letterpaper}                   		% ... or a4paper or a5paper or ... 
%\geometry{landscape}                		% Activate for rotated page geometry
%\usepackage[parfill]{parskip}    		% Activate to begin paragraphs with an empty line rather than an indent
\usepackage{graphicx}				% Use pdf, png, jpg, or eps§ with pdflatex; use eps in DVI mode
								% TeX will automatically convert eps --> pdf in pdflatex		

\usepackage{amsmath,amsfonts,amssymb}
\usepackage{tensor}

%SetFonts


\title{Isometry Groups}
\author{Michael Gehricke}
%\date{}							% Activate to display a given date or no date

\begin{document}
\maketitle
%\section{}
%\subsection{}
\newcommand{\vect}[1]{\mathbf{#1}}
\newcommand{\tenmix}[3]{{#1}^{#2}_{#3}}
%\equiv


\section{}
\subsection{}
Consider an inner product (.,.) of two elements of a vector space V with Basis B = $\vect{e_{i}}$. \\
\begin{equation}
g_{ij} = \vect{e}_{i} \cdot \vect{e}_{j}
\end{equation}
%
Linearly combining the $\vect{e}_{i}$ leads to a new Basis B'. Note: Over same indices upper and lower has to be summed (Einstein sum convention).
\begin{equation}
	\vect{e'}_{i} = A\indices{^k_i} \vect{e}_{k}	
\end{equation}
%
Substituting (2) into (1) leads to
\begin{equation}
	g_{ij}' = A\indices{^k_i} \vect{e}_{k} \cdot A\indices{^l_j} \vect{e}_{l} = A\indices{^k_i} \cdot A\indices{^l_j} \cdot g_{kl}
\end{equation}
%
Now suppose we want the transformation not to effect the metric $g_{ij}$ 
\begin{equation}
	g_{ij} = g'_{ij} \quad \implies A\indices{^k_l} \cdot A\indices{^l_m} = \delta^k_m \quad \text{, with} \quad \delta^i_j \equiv
	\begin{cases}
		1, & \text{if $i=j$}\\
        0, & \text{otherwise}
	\end{cases}
\end{equation}
%
In matrix notation this looks like $A^tA = I$. The transpose of matrix A times itself is the identity matrix. This result was obtained by choosing a bilinear metric. If the field of scalars are not the reals but complex or quaternion there is another possibility to set a metric for V: the sesquilinear metric. This then leads to the condition $A^\dagger A = 1$. That is the conjugate transpose of A times A is the identity matrix.
%

\subsection{The Unitary Group U(2,$\mathbb{C}$)}
For example let's take a general complex $2\times2$-matrix:
\begin{equation} A =
\begin{bmatrix}
	a & b \\
	c & d
\end{bmatrix} =
\begin{bmatrix}
	a_r + ia_i & b_r + ib_i \\
	c_r + ic_i & d_r + id_i
\end{bmatrix}
\end{equation} \\
Note: the last notation is separating the real and the imaginary part of a, b, c, d.
\pagebreak
\\
Applying the metric preserving condition $A^\dagger = A^{-1}$ to these matrices we obtain
\begin{equation} 
\begin{bmatrix}
	a_r - ia_i & c_r - ic_i \\
	b_r - ib_i & d_r - id_i
\end{bmatrix} =
\begin{bmatrix}
	 d_r + id_i & -b_r - ib_i \\
	-c_r - ic_i &  a_r + ia_i
\end{bmatrix}
\end{equation}
%
This leads to the general form of a matrix meeting this condition 
\begin{equation} A \in U(2,\mathbb{C}), \quad \text{A} = 
\begin{bmatrix}
	a_r - ia_i & -b_r - ib_i \\
	b_r - ib_i &  a_r + ia_i
\end{bmatrix}
\end{equation}
%
Renaming $a_r \rightarrow a_0$, $b_r \rightarrow a_1$, $a_i \rightarrow a_2$, $b_i \rightarrow a_3$ and setting $a_i = 0$ for all $i\neq j$ and $a_i = a_j$ for $i = j$ with $i,j \in \{0,1,2,3\}$, we can get a coordinate form of the above A.
%
%
\begin{equation}
A(a) = a_{0}
\begin{bmatrix}
	1 & 0 \\
	0 & 1
\end{bmatrix}
+a_{1}
\begin{bmatrix}
	0 & -1 \\
	1 & 0
\end{bmatrix}
+a_{2}
\begin{bmatrix}
	-i & 0 \\
	0 & i
\end{bmatrix}
+a_{3}
\begin{bmatrix}
	0 & -i \\
	-i & 0
\end{bmatrix}
\end{equation}
\\
%
Setting the symbols $\vect{e}$ (identity), $\vect{i}$, $\vect{j}$ and $\vect{k}$ for our basis vectors
\begin{equation}
\vect{e_0} \equiv
\begin{bmatrix}
	1 & 0 \\
	0 & 1
\end{bmatrix}
,\quad
\vect{e_1} \equiv
\begin{bmatrix}
	0 & -i \\
	-i & 0
\end{bmatrix}
,\quad
\vect{e_2} \equiv
\begin{bmatrix}
	0 & -i \\
	-i & 0
\end{bmatrix}
,\quad
\vect{e_3} \equiv
\begin{bmatrix}
	0 & -i \\
	-i & 0
\end{bmatrix}
\end{equation}
%
we can do pairwise products of this four matrices by following matrix multiplication rules and find \\
\begin{enumerate}
  \item $\vect{e_1} \times \vect{e_2} = \vect{e_3}$, $\vect{e_2} \times \vect{e_3} = \vect{e_1}$, $\vect{e_3} \times \vect{e_1} = \vect{e_2}$
  \item $\vect{e_2} \times \vect{e_1} = -\vect{e_3}$, $\vect{e_3} \times \vect{e_2} = -\vect{e_1}$, $\vect{e_1} \times \vect{e_3} = -\vect{e_2}$
  \item $\vect{e_1}^2 = \vect{e_2}^2 = \vect{e_3}^2 = -\vect{e_0}$
\end{enumerate}
%
As an overview here is the resulting multiplication table:
\begin{equation}
\begin{matrix}
	\vect{e_0} &  \vect{e_1} &  \vect{e_2} &  \vect{e_3} \\
	\vect{e_1} & -\vect{e_0} &  \vect{e_3} & -\vect{e_2} \\
	\vect{e_2} & -\vect{e_3} & -\vect{e_0} &  \vect{e_1} \\
	\vect{e_3} &  \vect{e_2} & -\vect{e_1} & -\vect{e_0} \\
\end{matrix}
\end{equation}
%
A product of to element $\vect{a} = a^i \vect{e}_i$ and $\vect{b} = b^j \vect{e}_j$ satisfies the following rule
\begin{subequations}
\begin{align*}
	\vect{a} &= a^i \vect{e_i} \cdot a^j \vect{e_j} \\
	&= (a_0b_0 - a_1b_1 - a_2b_2 - a_3b_3) \vect{e_0} \\
	&+ (a_1b_0 - a_0b_1 + a_2b_2 - a_3b_3) \vect{e_1} \\
	&+ (a_2b_0 - a_1b_1 + a_2b_2 - a_3b_3) \vect{e_2} \\
	&+ (a_3b_0 + a_1b_2 - a_2b_1 - a_3b_3) \vect{e_3} 
\end{align*}
\end{subequations}


%
\begin{subequations}
Maxwell's equations:
\begin{align}
        B'&=-\nabla \times E,\\
        E'&=\nabla \times B - 4\pi j,
\end{align}
\end{subequations}
%
\begin{equation} \vect{A\indices{^ij}} =
\begin{bmatrix}
	1 & 2 & 3 \\
	3 & 4 & 5 \\
	7 & 8 & 9
\end{bmatrix}
\end{equation}
%


   \begin{matrix} % or pmatrix or bmatrix or Bmatrix or ...
      a & b \\
      c & d \\
   \end{matrix}
%

\end{document}  